Ce projet est d�velopp� dans le cadre du mini-projet robotique en quatri�me ann�e du d�partement informatique. Le but du projet est de pratiquer les connaissances que nous avons acquises durant les s�ances du cours Robotique. 


Pour cela, une dizaine de mini-projets sont propos�es, et sont distribu�s aux autant de bin�mes. Nous avons obtenu ce sujet qui consiste des travails autour un algorithme: "Algorithme � Approximation Incr�mentale"(IAA). �tant donn� un bras manipulateur et les coordonn�es d'un point dans l'espace, cet algorithme sert � v�rifier si ce point est atteignable par ce bras. Et si oui, il peut trouver au moins un ensemble des valeurs articulaires possibles du bras pour que son organe terminal puisse atteindre ce point.


Nous avons commenc� notre travail par l'�tude de l'article\cite{1}, qui a pr�sent� cet algorithme ainsi que l'arri�re-plan du probl�me concern�. Apr�s avoir compris cet algorithme, nous l'avons impl�ment� en C++, et l'avons test� en utilisant une configuration d'un bras manipulateur donn� en TD.
