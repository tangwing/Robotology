
\documentclass[twoside]{EPURapport}
\usepackage{listings}
\usepackage[french]{algorithm2e}
%\renewcommand{\lstlistlistingname}{Liste des codes}
%\renewcommand{\lstlistingname}{Code}

%\addextratables{%
%	\lstlistoflistings
%}

%\swapAuthorsAndSupervisors



\nolistoftables
\thedocument{Rapport de mini-projet Robotique}{Test de l'accessibilit� d'un point par un bras manipulateur}

\grade{D�partement Informatique\\ 4\ieme{} ann�e\\ 2012 - 2013}

\authors{%
	\category{�tudiants}{%
		\name{Zheng LIU} \mail{zheng.liu@etu.univ-tours.fr}
		\name{Lei SHANG} \mail{lei.shang@etu.univ-tours.fr}
	}
	\details{DI4 2012 - 2013}
}

\supervisors{%
	\category{Encadrants}{%
		\name{Pierre Gaucher} \mail{pierre.gaucher@univ-tours.fr}
	}
	\details{Universit� Fran�ois-Rabelais, Tours}
}

\abstracts{Ce mini projet s'agit d'�tudier, puis impl�menter et tester l'Algorithme � Approximation Incr�mentale (IAA), qui est une m�thode permetant de v�rifier si un point situ� dans l'espace est atteignable par un bras manipulateur. Nous avons r�alis� cet algorithme en C++.}
{bras manipulateur, accessibilit� d'un point}
{This project is about our study of the Incremental Approximation Algorithm(IAA), which is a method telling whether a point in the space is reachable by a robot arm. We have studied, implemented and tested this method using C++.}
{robot arm, accessibility of a point}

\begin{document}

\chapter{Introduction}
Ce projet est d�velopp� dans le cadre du mini-projet robotique en quatri�me ann�e du d�partement informatique. Le but du projet est de pratiquer les connaissances que nous avons acquises durant les s�ances du cours Robotique. 


Pour cela, dizaine mini-projets sont propos�s, et sont distribu�s aux autant de bin�mes. Nous avons obtenu ce sujet qui consiste des travails autour un algorithme: "Algorithme � Approximation Incr�mentale"(IAA). Etant donn� un bras manipulateur et les coordonn�es d'un point dans l'espace, cet algorithme sert � v�rifier si ce point est atteignable par ce bras. Et si oui, il peut trouver au moins un ensemble des valeurs articulaires possibles du bras pour que son organe terminale puisse atteindre ce point.


Nous avons commenc� notre travail par l'�tude de l'article\cite{1}, qui a pr�sent� cet algorithme ainsi que l'arri�re-plan du probl�me concern�. Apr�s comprendre cet algorithme, nous avons l'impl�ment� en C++, et l'avons test� en utilisant une configuration d'un bras manipulateur donn� en TD.


\chapter{Pr�sentation de l'algorithme IAA}
\section{Probl�me r�el concern�} 
Apr�s obtenir ce sujet, nous avons tout d'abord �tudi� l'article\cite{1} propos� par notre encadrant. Nous nous apercevons que le sujet de notre projet est une mod�lisation de plein de probl�mes r�els. Dans l'article nous avons lu, le probl�me concret est d'�valuer l'accessibilit� dans un environnement b�ti lors que les capacit�s physique de la personne ne correspondent plus aux n�cessit� de l'habitat ce qui suivient en g�n�ral apr�s un accident de la vie. 


Ce probl�me, selon les auteurs de cet article, peut alors formul� comme un probl�me de r�solution de l'inverse cin�matique d'une structure articul�e compos�e de la personne et de son dispositif de mobilit� en consid�rant l'amplitude de ses capacit�s physiques r�siduelles.


\section{Cat�gories des solutions existantes}
Trois types de m�thodes sont disponible actuellement pour �tablir l'existence d'une inverse cin�matique de la cha�ne articulaire par rapport au point � atteindre:

\begin{itemize}
	\item Les m�thodes analytiques.
 	\item Les m�thodes de lin�arisation.
 	\item Les m�thodes d'optimisation.
\end{itemize}
\bigskip

Le troisi�me cat�gorie est le plus int�ressant et souvent utilis�e lorsque le nombre de varialbles est important et si l'on d�sire obtenir des solutions respectant certains crit�res. Le principe consiste � formuler le probl�me comme un probl�me d'optimisation par minimisation d'une fonction de c�ut. Ici la fonction de c�ut est de forme:
	\[
	\epsilon=\left|f([\Theta])-[X]\right|
	\]


Dans cette fonction:


\begin{itemize}
	\item $\Theta$ repr�sente l'ensemble des variables articulaires qu'on cherche.
 	\item $f([\Theta])$ repr�sente le point qu'on peut atteindre avec les variables $\Theta$.
 	\item $[X]$ repr�sente le point qu'on veut tester l'accessibilit�.
 	\item $\epsilon$ est donc la distance entre le point cible et le point qu'on peut atteindre pour le moment, et donc la valeur de $\epsilon$ est � minimiser.
\end{itemize}
\bigskip


Si on peut obtenir, � la fin de l'algorithme, un ensemble de $\Theta$ qui peut assurer que la valeur de $\epsilon$ est inf�rieur � une valeur pr�d�fini, alors on peut dire que le point cible est atteignable avec les variables $\Theta$. Et l'algorithme doit aussi traiter correctment le cas o� le point n'est pas atteignable.


Dans le probl�me r�el, il faut aussi prendre en compte que les variables articulaires qu'on obtient sont bien raisonnable. Par exemple, le joint coude humain n'a (normalement) pas de degr�e de libert� entre $[0, 2\pi]$.


Dans la section suivante, on va pr�senter l'algorithme IAA, qui nous permet de r�soudre ce probl�me.


\section{Algorithme IAA}
L'algorithme IAA\cite{1} est repr�sent� ci-dessous.


\begin{algorithm}[H]
 \SetAlgoLined
 \LinesNumbered
 Initialiser les variables $\Theta_i$ de mani�re al�atoire\;
 \Repeat{les condition d'arr�ts sont v�rifi�es}{
D�finir l'incr�ment Inc(i) (Incr�ment par variable articulaire)\;
		\ForEach{variable $\Theta_i$}{
				$\Theta_i = \Theta_i + Inc(i)$\;
				Calculer la distance entre la position courante et le but tel que $\epsilon=\left|f([\Theta])-[X]\right|$\;
				\eIf{$(\Delta\epsilon < 0)$}{ garder $\Theta_i$ (On est plus proche que le point cible)}
				{$\Theta_i = \Theta_i - 2*Inc(i)$ (chercher vers l'autre sens)\;
						Calculer $\epsilon=\left|f([\Theta])-[X]\right|$\;
						\eIf{$(\Delta\epsilon < 0)$}{garder $\Theta_i$}
						{$\Theta_i = \Theta_i + Inc(i)$ (garder la valeur d'origine)}}}
}
\end{algorithm}


Selon l'article \cite{2}, nous avons pu trouver une d�finition possible de la fonction $Inc(i)$:
\[
Inc(i) = (max(i) - min(i))*IncrementRate
\]
avec max(i) et min(i) sont les limites maximales et minimales de l'articulation i. La valeur de \textsl{IncrementRate} ajuste la vitesse de convergence de l'algorithme. La convergence est rapide au d�part et ensuite devient plus faible � proximit� de la solution. Une modification de \textsl{IncrementRate} est aussi propos�e:
\[
\textbf{If}(\Delta\epsilon = 0)  \textbf{Alors}  \textsl{IncrementRate} = \textsl{IncrementRate}/2
\]


\chapter{Impl�mentation de l'algorithme IAA}
%\section{Compl�ment de l'algorithme}
La description de l'algorithme IAA qu'on a pu trouver n'est pas pr�cise. Pour impl�menter cet algorithme, il nous reste du travail � faire pour pr�ciser tout l'algorithme. Le plus important est de fixer la condition d'arr�t de cet algorithme.


Pour cela, on consid�re deux cas possible que notre algorithme rencontrera:
\bigskip
\begin{enumerate}
	\item Le point cible est atteignable
	\item Le point cible n'est pas atteignable
\end{enumerate}


\subsection{Premier cas}
Pour le premier cas, l'algorithme doit s'arr�ter quand la distance entre le point courant et le point cible est inf�rieur � une valeur pr�d�finie (0.00001 par exemple), et il doit renvoyer l'ensemble des variable articulaires courantes.


La d�finition de la valeur de borne est tr�s souple selon le probl�me r�el. Par exemple dans le probl�me pr�sent� dans le chapitre pr�c�dant, la pr�cision n'est pas trop importante, car on peut dire sans probl�me que "une personne peut atteidre un point dans sa chambre si la distance entre ce point et le bout de son doigt est seulement 1mm". Dans notre impl�mentation, cette valeur est 0.0001, qui est � la fois assez petite et pas trop co�teuse par rapport au temp d'execution essentiel pour arriver � la fin de l'algorithme.


\subsection{Deuxieme cas}
Pour le deuxi�me cas, puisque le point cible n'est r�elment pas atteignable, on ne va jamais avoir une distance assez petite entre le point courant et le point cible. En analysant l'algorithme, nous pensons intuitivement que l'algorithme doit s'arr�ter quand il arrive un �tat "consistant". Ca veut dire que l'algorithme ne peut d�j� pas am�liorer aucune variable articulaire $\Theta_i$ dans deux it�rations cons�cutives. 


Nous avons test� notre algorithme avec cette condition d'arr�t, mais le r�sultat n'est pas pr�f�rable. Apr�s une analyse plus d�taill�e, nous avons pu trouver le truc : m�me si le point cible est atteignable, on peut aussi avoir des �tat consistant pendant l'execution de l'algorithme � cause de la valeur de \textsl{IncrementRate}. C'est assez raisonnable, car quand le changement est trop grand, c'est possible que aucun changement n'est utile pour am�liorer le r�sultat, et c'est pour �a qu'on diminue la valeur \textsl{IncrementRate} en ce moment-l� en apportant des changement plus d�licats. Et donc, on ajoute une autre condition en dehors de la consistance d'�tat, c'est que la valeur de \textsl{IncrementRate} doit �tre assez petite pour assurer qu'il n'y a vraiment pas de possibilit� d'am�liorer le r�sultat.

BORNE[i]=0.000001/((QUAconfig[i].maxTheta - QUAconfig[i].minTheta)*180/PI);//precision:0.00...1�
%%%%%%%
En un mot, l'algorithme IAA (avec notre impl�mentation) doit s'arr�ter quand l'une des deux conditions ci-apr�s est v�rifier: soit la distance entre le point courant et le point cible est inf�rieur � une valeur pr�d�finie; soit la valeur de \textsl{IncrementRate} est inf�rieur � une valeur pr�d�finie et les variables articulaires $\Theta$ ont un �tat consistant.


\section{Impl�mentation de l'algorithme IAA}
\subsection{Choix de langage}
Nous avons choisi C++ en tant que langage de programmation avec les consid�ration ci-apr�s:
\begin{itemize}
	\item Ce travail consiste � plut�t les calculs au lieu des logique commercials ou interface de interation. C'est tr�s efficace d'utiliser C++ que les autre langages sup�rieurs.
	\item C++ poss�de des carat�ristiques orient�-objet, qui peut nous faciliter
\end{itemize}} 


\section{Test de l'algorithme IAA}
%
\chapter{Test de l'algorithme IAA}
%\section{Probl�mes r�solus}
\subsection{Probl�me sur la pr�cision}
Au d�but de l'impl�mentation de l'algorithme, nous obtenions souvent des r�sultats d�raisonnables. Apr�s avoir discut� avec notre encadrant, nous nous sommes aper�us que �a soit un probl�me de pr�cision. �a concerne la repr�sentation interne des donn�es en C++. Et puis nous avons chang� notre type de donn�e � \textbf{double} qui a enfin r�solu ce probl�me.


\section{Probl�mes non r�solus}
\subsection{Probl�me des minimums locaux}
Un avantage de l'algorithme IAA est qu'on peut imposer des limits sur les variables articulaires, pour que la solution respecte certains crit�res particuliers. Mais �a nous am�ne aussi le probl�me des minimums locaux. Parfois l'algorithme se converge vers un minimum local (mais pas global), et donc dans un �tat consistant. En cons�quence, il nous dit qu'il y a pas de solution (le point cible n'est pas atteignable), mais le point est en fait atteignable, et la solution n'�tait pas trouv�e � cause des minimums locaux.


C'est un probl�me commun des algorithmes similaires. C'est � cause des limites qu'on a impos�es aux variables articulaires et � la initialisation al�atoire d'elles. Nous ne pouvons pas �liminer ce probl�me, mais nous pouvons essayer de l'examiner, pour qu'on puisse relancer l'algorithme au lieu de croire qu'il n'y a pas de solution.


Nous constatons que quand un "minimum local" appara�t, il y a au moins une variable articulaire qui a atteint sa limite qu'on l'a impos�e. Selon �a, dans notre programme, si aucune solution n'est trouv�e, on va tester les variables articulaires courantes, s'il existe une variable articulaire qui a atteint sa limite, alors on affiche un message disant qu'on a \textbf{probablement} rencontr� un minimum local. � savoir qu'on ne l'a pas prouv� mais plut�t le tester selon les exp�riences.

\begin{figure}[!ht]
	\centering
		\includegraphics[scale=0.4]{pics/ch3_localminimum.png}
	\caption{Message affich� quand un minimum local appara�t}
	\label{fig:ch3_localminimum}
\end{figure}

\subsection{Probl�me de vitesse de convergence}
La vitesse de convergence est aussi un probl�me de cet algorithme, surtout quand le point cible est sur la fronti�re de la zone atteignable du bras manipulateur. En ce cas particulier, l'algorithme peut enfin trouver la solution, mais la vitesse de convergence devient tr�s lente quand l'organe terminal approche du point cible.



%
%\chapter{Mise en oeuvre}
%\input{Ch4.tex}
%
\chapter{Difficult�s rencontr�es}
%\input{Ch5.tex}
%
\chapter{Conclusion}
%Notre projet concerne la compr�hension, impl�mentation et le test d'un l'algorithme(IAA) qui peut v�rifier l'accessibilit� d'un point dans l'espace par un bras manipulateur. Nous commencions notre travail par la lecture de deux articles propos�s par notre encadrant. Apr�s avoir compris l'algorithme, nous l'avons impl�ment� en C++, et l'avons test� avec quelques exemples de bras manipulateur qu'on a vu en cours.


Ce projet nous a renforc� la compr�hension sur la cin�matique robotique, surtout le \textit{Mod�le G�om�trique Direct} et le \textit{Mod�le G�om�trique Inverse}. �a nous permet aussi d'avoir une impression sur l'application de robotique dans la vie r�elle.

\begin{thebibliography}{9}
   \bibitem{1}
          Otmani R., 
          Pruski A.,
          Belarbi K.,
          \emph{La r�alit� virtuelle comme outil pour l'�valuation, la visualisation et la validation de l'accessibilit� d'un lieu de vie}.
          Conf�rence Handicap 2010,
          juin 2010,
          Paris.
    \bibitem{2}
    			Abdelhak MOUSSAOUI,
    			\emph{Prise en charge de psychotherapie et du handicap par la r�alit� virtuelle}
\end{thebibliography}
\annexes

\end{document}

